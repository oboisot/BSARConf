%\documentclass[tikz,border=0pt]{standalone}
\documentclass[multi=tikzpicture,varwidth=false]{standalone} % Permet d'éviter le clash d'import du package xcolor avec tikz
% qui doit être importé avant tikz (pas possible sir tikz est déclaré dans le préambule de la classe standalone)

% Pour les symboles et équations mathématiques
\usepackage{amsmath,amssymb,amsfonts}
\usepackage{textcomp} % Pour la commande \textdegree
% Couleurs
\usepackage[dvipsnames]{xcolor}
% Dessins en LaTeX
\usepackage{tikz}
\usepackage{tikz-3dplot}
\usetikzlibrary{backgrounds}
%\usetikzlibrary{positioning}

%%% Définition de COMMANDES PERSO %%%
% Vectors
\newcommand{\vecl}[1]{\overrightarrow{#1}} % Vector
\newcommand{\vecn}[1]{\widehat{#1}} % normalized vector
\newcommand{\norm}[2][]{\left\Vert\vecl{#2}#1\right\Vert}
% Parameters
\newcommand{\PRI}{\mathrm{PRI}}
\newcommand{\PRF}{\mathrm{PRF}}
\newcommand{\hatt}{\hat{t}}
\newcommand{\hatto}{\widehat{t_0}}
\newcommand{\hattT}{\widehat{t_T}}
\newcommand{\hattR}{\widehat{t_R}}
\newcommand{\bigdot}[1]{\overset{\hbox{\tiny$\bullet$}}{#1}}
\newcommand{\bigdotdot}[1]{\overset{\hbox{\tiny$\bullet$$\bullet$}}{#1}}
\newcommand{\conj}[1]{\overline{#1}}

\definecolor{backgroundcolor}{HTML}{8b8989}
\definecolor{Pcolor}{HTML}{ffa500}
\definecolor{gridcolor}{HTML}{536878}

\begin{document}
\tdplotsetmaincoords{60}{40}
        %\begin{tikzpicture}[tdplot_main_coords,scale=1.5,background rectangle/.style={fill=backgroundcolor}, show background rectangle]
        \begin{tikzpicture}[tdplot_main_coords,scale=1.25]
            \pgfmathsetmacro{\Txx}{-2};\pgfmathsetmacro{\Txy}{-3};\pgfmathsetmacro{\Txz}{4}; % Tx
            \pgfmathsetmacro{\Rxx}{1};\pgfmathsetmacro{\Rxy}{4};\pgfmathsetmacro{\Rxz}{2}; % Rx
            \pgfmathsetmacro{\Px}{1.5};\pgfmathsetmacro{\Py}{1};\pgfmathsetmacro{\Pz}{0}; % P
            \pgfmathsetmacro{\utx}{0.52615222};\pgfmathsetmacro{\uty}{0.60131682};\pgfmathsetmacro{\utz}{-0.60131682}; % utx
            \pgfmathsetmacro{\urx}{0.13736056};\pgfmathsetmacro{\ury}{-0.82416338};\pgfmathsetmacro{\urz}{-0.54944226}; % urx
            \pgfmathsetmacro{\ueffx}{0.33175639};\pgfmathsetmacro{\ueffy}{-0.11142328};\pgfmathsetmacro{\ueffz}{-0.57537954}; % ueff
            \pgfmathsetmacro{\omegx}{-0.24618911};\pgfmathsetmacro{\omegy}{0.5858917};\pgfmathsetmacro{\omegz}{-0.07174075}; % omegaeff
            \pgfmathsetmacro{\Bx}{-0.06102167};\pgfmathsetmacro{\By}{1.52428276};\pgfmathsetmacro{\Bz}{2.70734778}; % B
            \pgfmathsetmacro{\vtx}{-1};\pgfmathsetmacro{\vty}{1};\pgfmathsetmacro{\vtz}{0}; % vtx
            \pgfmathsetmacro{\vrx}{0.5};\pgfmathsetmacro{\vry}{0.5};\pgfmathsetmacro{\vrz}{0}; % vrx
            \coordinate (O) at (0,0,0);
            \coordinate (Tx) at (\Txx,\Txy,\Txz);
            \coordinate (Rx) at (\Rxx,\Rxy,\Rxz);
            \coordinate (P) at (\Px,\Py,\Pz);
            \coordinate (B) at (\Bx,\By,\Bz);
            % Repère terrestre
            \draw[gridcolor] (\Px-4,\Py-4,0) -- (\Px+4,\Py-4,0) -- (\Px+4,\Py+4,0) -- (\Px-4,\Py+4,0) -- cycle;
            \foreach \i in {-3,...,3} {
       		 \draw [very thin,gridcolor] (\Px-4,\Py-\i,0) -- (\Px+4,\Py-\i,0);
       		 \draw [very thin,gridcolor] (\Px-\i,\Py-4,0) -- (\Px-\i,\Py+4,0);
    		}
            \draw[-stealth,thick,red] (P) -- (\Px+1,\Py,\Pz) node[right]{$\hat{e}$};
            \draw[-stealth,thick,green] (P) -- (\Px,\Py+1,\Pz) node[above right]{$\hat{n}$};
            \draw[-stealth,thick,blue] (P) -- (\Px,\Py,\Pz+1) node[anchor=east]{$\hat{u}$};
            % Segments TxP, RxP, TxRx
            \draw[ultra thin] (Tx) -- (P) -- (Rx);
            \draw[dashed,ultra thin] (Tx) -- (Rx);
            % Tracés des points Tx, Rx et P
            \draw[color=white,fill=white] (Tx) circle (0.06cm);\draw[] (\Txx,\Txy,\Txz+0.1) node[left,white]{$T(u_0)$}; % Tx
            \draw[color=black,fill=black] (Rx) circle (0.06cm);\draw[] (\Rxx,\Rxy,\Rxz+0.1) node[above,black]{$R(u_0)$}; % Tx
            \draw[color=Pcolor,fill=Pcolor] (P) circle (0.06cm);\draw[] (\Px,\Py,\Pz) node[Pcolor,below left]{$P$}; % P
            % Vecteur de visée et vitesse
            \draw[thick,-latex,white] (Tx) -- (\Txx+\utx,\Txy+\uty,\Txz+\utz) node[white,below left] {$\widehat{u_{TP}}(u_0)$};
            \draw[thick,-latex,white] (Tx) -- (\Txx+\vtx,\Txy+\vty,\Txz+\vtz) node[white,left] {$\vecl{V_{T}}(u_0)$};
            \draw[thick,-latex,black] (Rx) -- (\Rxx+\urx,\Rxy+\ury,\Rxz+\urz) node[black,below right] {$\widehat{u_{RP}}(u_0)$};
            \draw[thick,-latex,black] (Rx) -- (\Rxx+\vrx,\Rxy+\vry,\Rxz+\vrz) node[black,right] {$\vecl{V_R}(u_0)$};
            % Tracé de ueff, omegaeff
            \draw[ultra thin, dashed] (B) -- (P);
            \draw[thick,-latex,Red] (B) -- (\Bx+\ueffx,\By+\ueffy,\Bz+\ueffz) node[right,Red] {$\vecl{\beta}(u_0)$};
            \draw[thick,-latex,NavyBlue] (B) -- (\Bx+\omegx,\By+\omegy,\Bz+\omegz) node[right,NavyBlue] {$\bigdot{\vecl{\beta}}(u_0)$};
            \draw[color=black,fill=black] (B) circle (0.06cm);
            % 
            \draw[-latex,Red] (P) -- (\Px+\ueffx,\Py+\ueffy,\Pz) node[below,Red] {$\vecl{\beta_{g}}(u_0)$};
            \draw[-latex,NavyBlue] (P) -- (\Px+\omegx,\Py+\omegy,\Pz); %node[left,NavyBlue] {$\bigdot{\vecl{\beta_{g}}}(u_0)$};
            \node[left,NavyBlue,right] at (\Px-1.2,\Py+1.3,\Pz) {$\bigdot{\vecl{\beta_{g}}}(u_0)$};
            % Tracés de l'angle bistatique
            \tdplotdefinepoints(\Px,\Py,\Pz)(\Txx,\Txy,\Txz)(\Rxx,\Rxy,\Rxz);
            \tdplotdrawpolytopearc[<->]{1.2}{above}{$\beta(u_0)$};
        \end{tikzpicture}
\end{document}