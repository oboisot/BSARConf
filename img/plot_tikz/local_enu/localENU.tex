\documentclass[tikz,border=0pt]{standalone}
%\documentclass[multi=tikzpicture,varwidth=false,9pt]{standalone} % Permet d'éviter le clash d'import du package xcolor avec tikz
% qui doit être importé avant tikz (pas possible sir tikz est déclaré dans le préambule de la classe standalone)

% Pour les symboles et équations mathématiques
\usepackage{amsmath,amssymb,amsfonts}
% Dessins en LaTeX
\usepackage{tikz}
\usepackage{tikz-3dplot}

% Colors
\definecolor{azure}{rgb}{0.0, 0.5, 1.0}
\definecolor{refgreen}{HTML}{006400}

%%%%%%%%%%%%%%%%%
% Ellipsoid Drawing Functions %
%%%%%%%%%%%%%%%%%
% Draw full meridian at a given longitude and altitude
\newcommand{\DrawMeridian}[5][black]{ % equatorial radius, polar radius, longitude,altitude.
							                 % Optional = drawing options
    % Function call :
    % \DrawMeridian[<drawing options>]{<equatorial radius>}{<polar radius>}{<longitude>}{<altitude>}
    \pgfmathsetmacro{\es}{1-#3*#3/(#2*#2)}; % eccentricity squared
    \pgfmathsetmacro{\clon}{cos(#4)};\pgfmathsetmacro{\slon}{sin(#4)}; % cos(lon0), sin(lon0)
    \draw[#1] ({(#2+#5)*\clon},{(#2+#5)*\slon},0)%
        \foreach \lat in {1,2,...,359} {%
            \pgfextra%
                \pgfmathsetmacro{\clat}{cos(\lat)};\pgfmathsetmacro{\slat}{sin(\lat)};% cos(lat), sin(lat)
                \pgfmathsetmacro{\nu}{#2/sqrt(1-\es*\slat*\slat)};% nu(lat)
                \pgfmathsetmacro{\x}{(\nu+#5)*\clat*\clon};%
                \pgfmathsetmacro{\y}{(\nu+#5)*\clat*\slon};%
                \pgfmathsetmacro{\z}{((1-\es)*\nu+#5)*\slat};%
           \endpgfextra%
            -- (\x, \y, \z)%
    } -- cycle;%
}%
% Draw partial meridian at a given longitude and altitude between latmin and latmax.
\newcommand{\DrawMeridianBetween}[7][black]{ % equatorial radius, polar radius, longitude,altitude,latmin,latmax.
							                                   % Optional = drawing options
    % Function call :
    % \DrawMeridianBetween[<drawing options>]{<equatorial radius>}{<polar radius>}{<longitude>}{<altitude>}{<latmin>}{<latmax>}
    \pgfmathsetmacro{\es}{1-#3*#3/(#2*#2)}; % eccentricity squared
    \pgfmathsetmacro{\clon}{cos(#4)};\pgfmathsetmacro{\slon}{sin(#4)}; % cos(lon0), sin(lon0)
    % Initial point
    \pgfmathsetmacro{\clatmin}{cos(#6)};\pgfmathsetmacro{\slatmin}{sin(#6)};% cos(lat), sin(lat)
    \pgfmathsetmacro{\nu}{#2/sqrt(1-\es*\slatmin*\slatmin)};% nu(lat)
    \pgfmathsetmacro{\x}{(\nu+#5)*\clatmin*\clon};%
    \pgfmathsetmacro{\y}{(\nu+#5)*\clatmin*\slon};%
    \pgfmathsetmacro{\z}{((1-\es)*\nu+#5)*\slatmin};%
    % bounds
    \pgfmathsetmacro{\latmin}{#6+1};\pgfmathsetmacro{\latmax}{#7};
    \pgfmathsetmacro{\dlat}{#6+2};
    \draw[#1] (\x,\y,\z)%
        \foreach \lat in {\latmin,\dlat,...,\latmax} {%
            \pgfextra%
                \pgfmathsetmacro{\clat}{cos(\lat)};\pgfmathsetmacro{\slat}{sin(\lat)};% cos(lat), sin(lat)
                \pgfmathsetmacro{\nu}{#2/sqrt(1-\es*\slat*\slat)};% nu(lat)
                \pgfmathsetmacro{\x}{(\nu+#5)*\clat*\clon};%
                \pgfmathsetmacro{\y}{(\nu+#5)*\clat*\slon};%
                \pgfmathsetmacro{\z}{((1-\es)*\nu+#5)*\slat};%
           \endpgfextra%
            -- (\x, \y, \z)%
    };%
}%
% Draw all full meridians at a given longitude and altitude spaced by delta lon
\newcommand{\DrawMeridians}[5][black]{ % equatorial radius, polar radius, delta lon, altitude.
									% Optional = drawing option
    % Function call :
    % \DrawMeridians[<drawing options>]{<equatorial radius>}{<polar radius>}{<delta lon>}{<altitude>}
    \pgfmathsetmacro{\es}{1-#3*#3/(#2*#2)}; % eccentricity squared
    \pgfmathsetmacro{\lmax}{360-#4};
    \foreach \lon in {0,{#4},...,\lmax} { % Meridians drawing
        \pgfmathsetmacro{\clon}{cos(\lon)};\pgfmathsetmacro{\slon}{sin(\lon)}; % cos(lon0), sin(lon0)
        \draw[#1] ({(#2+#5)*\clon},{(#2+#5)*\slon},0)%
            \foreach \lat in {1,2,...,359} {%
                \pgfextra%
                    \pgfmathsetmacro{\clat}{cos(\lat)};\pgfmathsetmacro{\slat}{sin(\lat)};% cos(lat), sin(lat)
                    \pgfmathsetmacro{\nu}{#2/sqrt(1-\es*\slat*\slat)};% nu(lat)
                    \pgfmathsetmacro{\x}{(\nu+#5)*\clat*\clon};%
                    \pgfmathsetmacro{\y}{(\nu+#5)*\clat*\slon};%
                    \pgfmathsetmacro{\z}{((1-\es)*\nu+#5)*\slat};%
               \endpgfextra%
                -- (\x, \y, \z)%
        } -- cycle;%
    }%
}

% Draw full parallel at a given latitude and altitude
\newcommand{\DrawParallel}[5][black]{ % equatorial radius, polar radius, latitude,altitude.
							                 % Optional = drawing options
    % Function call :
    % \DrawParallel[<drawing options>]{<equatorial radius>}{<polar radius>}{<latitude>}{<altitude>}
    \pgfmathsetmacro{\es}{1-#3*#3/(#2*#2)}; % eccentricity squared
    \pgfmathsetmacro{\clat}{cos(#4)};\pgfmathsetmacro{\slat}{sin(#4)}; % cos(phi0), sin(phi0)
    \pgfmathsetmacro{\nu}{#2/sqrt(1-\es*\slat*\slat)}; % nu(phi0)
    \draw[#1] ({(\nu+#5)*\clat},0,{((1-\es)*\nu+#5)*\slat})%
        \foreach \lon in {1,2,...,359} {%
            \pgfextra%
                \pgfmathsetmacro{\x}{(\nu+#5)*\clat*cos(\lon)};%
                \pgfmathsetmacro{\y}{(\nu+#5)*\clat*sin(\lon)};%
                \pgfmathsetmacro{\z}{((1-\es)*\nu+#5)*\slat};%
           \endpgfextra%
            -- (\x, \y, \z)%
    } -- cycle;%
}%
% Draw partial parallel at a given latitude and altitude between lonmin and lonmax
\newcommand{\DrawParallelBetween}[7][black]{ % equatorial radius, polar radius, latitude,altitude,lonmin,lonmax.
							                                % Optional = drawing options
    % Function call :
    % \DrawParallelBetween[<drawing options>]{<equatorial radius>}{<polar radius>}{<latitude>}{<altitude>}{<lonmin>}{<lonmax>}
    \pgfmathsetmacro{\es}{1-#3*#3/(#2*#2)}; % eccentricity squared
    \pgfmathsetmacro{\clat}{cos(#4)};\pgfmathsetmacro{\slat}{sin(#4)}; % cos(phi0), sin(phi0)
    \pgfmathsetmacro{\nu}{#2/sqrt(1-\es*\slat*\slat)}; % nu(phi0)
     % Initial point
    \pgfmathsetmacro{\clonmin}{cos(#6)};\pgfmathsetmacro{\slonmin}{sin(#6)};% cos(lat), sin(lat)
    \pgfmathsetmacro{\x}{(\nu+#5)*\clat*\clonmin};%
    \pgfmathsetmacro{\y}{(\nu+#5)*\clat*\slonmin};%
    \pgfmathsetmacro{\z}{((1-\es)*\nu+#5)*\slat};%
    % bounds
    \pgfmathsetmacro{\lonmin}{#6+1};\pgfmathsetmacro{\lonmax}{#7};
    \pgfmathsetmacro{\dlon}{#6+2};
    \draw[#1] (\x,\y,\z)%
        \foreach \lon in {\lonmin,\dlon,...,\lonmax} {%
            \pgfextra%
                \pgfmathsetmacro{\x}{(\nu+#5)*\clat*cos(\lon)};%
                \pgfmathsetmacro{\y}{(\nu+#5)*\clat*sin(\lon)};%
                \pgfmathsetmacro{\z}{((1-\es)*\nu+#5)*\slat};%
           \endpgfextra%
            -- (\x, \y, \z)%
    };%
}%

% Fonction de tracé d'un méridien à la longitude et l'altitude souhaitées
\newcommand{\DrawParallels}[5][black]{ % equatorial radius, polar radius, delta lat, altitude.
							                 % Optional = drawing options
    % Function call :
    % \DrawParallels[<drawing options>]{<equatorial radius>}{<polar radius>}{<delta lat>}{<altitude>}
    \pgfmathsetmacro{\es}{1-#3*#3/(#2*#2)}; % eccentricity squared
    \pgfmathsetmacro{\latmin}{-90+#4};\pgfmathsetmacro{\latmax}{90-#4};\pgfmathsetmacro{\dlat}{\latmin+#4};
    \foreach \lat in {\latmin,\dlat,...,\latmax} {
    \pgfmathsetmacro{\clat}{cos(\lat)};\pgfmathsetmacro{\slat}{sin(\lat)}; % cos(phi0), sin(phi0)
        \pgfmathsetmacro{\nu}{#2/sqrt(1-\es*\slat*\slat)}; % nu(phi0)
        \draw[#1] ({(\nu+#5)*\clat},0,{((1-\es)*\nu+#5)*\slat})%
            \foreach \lon in {1,2,...,359} {%
                \pgfextra%
                    \pgfmathsetmacro{\x}{(\nu+#5)*\clat*cos(\lon)};%
                    \pgfmathsetmacro{\y}{(\nu+#5)*\clat*sin(\lon)};%
                    \pgfmathsetmacro{\z}{((1-\es)*\nu+#5)*\slat};%
               \endpgfextra%
                -- (\x, \y, \z)%
        } -- cycle;%
    }
}%

% Ellipsoid drawing function
\newcommand{\DrawEllipsoid}[5][black]{
     % Function call :
    % \DrawParallels[<drawing options>]{<equatorial radius>}{<polar radius>}{<delta lat/lon>}{<altitude>}
    \DrawMeridians[#1]{#2}{#3}{#4}{#5};
    \DrawParallels[#1]{#2}{#3}{#4}{#5};
}

\begin{document}
\pagecolor{black}
% sqare of half axes
% view angle
\tdplotsetmaincoords{70}{130}
%
\begin{tikzpicture}[tdplot_main_coords,scale=3]
    % Tracé de l'ellipsoïde
        % Ellipsoid definition
    \pgfmathsetmacro{\a}{2}; % a
    \pgfmathsetmacro{\b}{1.5}; % b
    \pgfmathsetmacro{\es}{1-\b*\b/(\a*\a)}; % e²
        % Point P definition
    \pgfmathsetmacro{\lat}{50}; % latitude [°]
    \pgfmathsetmacro{\lon}{60}; % longitude [°]
    \pgfmathsetmacro{\alt}{.5}; % altitude [m]
    % Tracé des axes de références
    \draw[white,thin,-stealth,thick] (0,0,0) -- ({\a+0.1},0,0) node[below left]{$\hat{x}$}; % x-axis
    \draw[white,thin,-stealth,thick] (0,0,0) -- (0,{\a+0.1},0) node[anchor=north west]{$\hat{y}$}; % y-axis
    \draw[white,thin,-stealth,thick] (0,0,-0.9) -- (0,0,{\b+0.1}) node[anchor=south]{$\hat{z}$}; % z-axis
    \node[white,opacity=1,above left] at (0,0,0) {$O$};

    % Tracé de l'ellipsoide + équateur + Greenwich meridian
    \DrawEllipsoid[azure,thin,opacity=0.2]{\a}{\b}{10}{0};
    \DrawParallelBetween[azure,thick,fill=white,fill opacity=0.2]{\a}{\b}{0}{0}{-50}{130};
    \DrawParallelBetween[azure,thick,densely dashed,fill=white,fill opacity=0.2]{\a}{\b}{0}{0}{130}{310};
    % Meridian and Parallel of point P at altitude 0
    \DrawMeridianBetween[azure,thin,fill=white,fill opacity=0.2]{\a}{\b}{\lon}{0}{-70}{110};
    \DrawMeridianBetween[azure,thin,densely dashed,fill=white,fill opacity=0.2]{\a}{\b}{\lon}{0}{110}{290};
    \DrawParallelBetween[azure,thin]{\a}{\b}{\lat}{0}{-70}{150};
    \DrawParallelBetween[azure,thin,densely dashed]{\a}{\b}{\lat}{0}{150}{290};
    % Meridian and Parallel of point P at altitude h
    \DrawParallel[refgreen,thin]{\a}{\b}{\lat}{\alt};
    \DrawMeridianBetween[refgreen,thin]{\a}{\b}{\lon}{\alt}{-105}{155};
    \DrawMeridianBetween[refgreen,thin,densely dashed]{\a}{\b}{\lon}{\alt}{155}{255};

    % Axes plan local ENU (h = 0.5)    				            % (h = 0)
    \coordinate (P) at (0.90627959, 1.56972229, 1.38264192);%(0.74558268, 1.29138709, 0.9996197);
    \coordinate (Psurf) at (0.74558268, 1.29138709, 0.9996197);
    \coordinate (Pe) at (0.68977324, 1.69472229, 1.38264192);%(0.52907633, 1.41638709, 0.9996197);
    \coordinate (Pn) at (0.81052403, 1.4038688 , 1.54333882);%(0.64982713, 1.1255336 , 1.1603166);
    \coordinate (Pu) at (0.98662804, 1.70888989, 1.57415303);%(0.82593113, 1.43055469, 1.19113081);
    \coordinate (nu) at (0, 0, -0.77748199); % nu(phi)
    % Plan ENU
    \draw[white,line width=.3pt,opacity=1,dashed] (P) -- (Psurf) node[midway,right] {\footnotesize$H$} -- (nu);
    \draw[azure,fill=azure,opacity=1] (Psurf) circle (0.015cm);
    \fill[refgreen,opacity=0.3]
         (0.28175577, 1.48801531, 1.70403572) -- (1.14778118, 0.98801531, 1.70403572) --
         (1.5308034 , 1.65142926, 1.06124811) -- (0.664778  , 2.15142926, 1.06124811) -- cycle;
    \draw[line width=1pt,refgreen,opacity=1]
         (0.28175577, 1.48801531, 1.70403572) -- (1.14778118, 0.98801531, 1.70403572) --
         (1.5308034 , 1.65142926, 1.06124811) -- (0.664778  , 2.15142926, 1.06124811) -- cycle;
    \draw[red,-stealth,thick,opacity=1] (P) -- (Pe) node[below]{$\hat{e}$}; % e
    \draw[green,-stealth,thick,opacity=1] (P) -- (Pn) node[left]{$\hat{n}$}; % n
    \draw[blue,-stealth,thick,opacity=1] (P) -- (Pu) node[above right]{$\hat{u}$}; % u
    \draw[refgreen,fill=refgreen] (P) circle (0.015cm) node[left] {$P$};
    % Tracés coordonnées lat, lon, h
         % lambda
    \draw[white,line width=.3pt,opacity=1,dashed] (0,0,0) -- (1, 1.73205081, 0) {};
    \tdplotdefinepoints(0,0,0)(2,0,0)(1, 1.73205081, 0);
    \tdplotdrawpolytopearc[white]{0.6}{below}{\color{white}\large$\lambda$};
        % phi
    \tdplotdefinepoints(0.32619242, 0.56498185, 0)(1, 1.73205081, 0)(0.64758623, 1.12165225, 0.76604444);
    \tdplotdrawpolytopearc[white]{0.5}{right}{\color{white}\large$\phi$};
\end{tikzpicture}
%
\end{document}
